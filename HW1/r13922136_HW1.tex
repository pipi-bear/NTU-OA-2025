\documentclass{article}
\usepackage[utf8]{inputenc}
\usepackage[english]{babel}
\usepackage[]{amsthm} 
\usepackage[]{amssymb} 
\usepackage{amsmath}
\usepackage{graphicx}
\usepackage{hyperref}
\usepackage{comment}
\usepackage[dvipsnames]{xcolor}
\usepackage[thinc]{esdiff}
\newtheorem*{theorem}{Theorem}

\title{Optimization Algorithms: HW1}
\author{Lo Chun, Chou \\ R13922136}
\date\today

\begin{document}
\setlength{\parindent}{0pt}
\maketitle 

\section*{1}

\section*{2}

Given a twice differentiable function $\varphi: \mathbb{R}^d \to [- \infty, \infty]$, 
assume that it is logarithmically homogeneous, 
then by the definition, the following holds:

\begin{align*}
    \varphi ( \gamma x ) = \varphi (x) - \log \gamma, \quad \forall x \in \mathbb{R}^d, \gamma > 0 \tag{1}
\end{align*}

\textcolor{blue}{\underline{Claim: $\langle \nabla \varphi (x), x \rangle = - 1$ } }
\bigskip

To derive the first equation, we first define the following:

\begin{align*}
    F(\gamma) = \textcolor{orange}{\varphi (\gamma x)}
\end{align*}

Then the original equation $(1)$ would become:

\begin{align*}
    F(\gamma) = \textcolor{Green}{\varphi (x) - \log \gamma}
\end{align*}

Taking the derivative w.r.t. $\gamma$ on both sides, we get:

\begin{align*}
    \frac{dF}{d\gamma} &= \frac{d}{d\gamma} \textcolor{orange}{\varphi (\gamma x)} = \nabla \varphi (\gamma x) \cdot x = \langle \nabla \varphi (\gamma x), x \rangle \tag{2}\\
    \frac{dF}{d\gamma} &= \frac{d}{d\gamma} (\textcolor{Green}{\varphi (x) - \log \gamma}) = - \frac{1}{\gamma} \tag{3}
\end{align*}

Thus by $(2)$ and $(3)$, we have:

\begin{align*}
    \langle \nabla \varphi (\gamma x), x \rangle = - \frac{1}{\gamma}
\end{align*}
Then by plugging in $\gamma = 1$, we have:

\begin{align*}
    \langle \nabla \varphi (x), x \rangle = - 1 \qquad \square
\end{align*}

\textcolor{blue}{\underline{Claim: $\nabla \varphi (x) = - \nabla^2 \varphi ( x ) x$ } }
\bigskip

From the previous part, we have:

\begin{align*}
    \nabla \varphi (x)^T x = - 1
\end{align*}

Compute the gradient of both sides, for the left hand side, we have:

\begin{align*}
    \textcolor{orange}{\nabla(}\nabla \varphi (x)^T x\textcolor{orange}{)}
    &= \textcolor{orange}{\nabla (}\nabla \varphi (x)\textcolor{orange}{)}^T x + \nabla \varphi (x)^T \textcolor{orange}{\nabla} x  \\
    &= \nabla^2 \varphi (x) x + \nabla \varphi (x)^T \nabla x  \\
\end{align*}

For the right hand side, we have:

\begin{align*}
    \nabla (- 1) = 0
\end{align*}

Thus we have:

\begin{align*}
    &\nabla^2 \varphi (x) x + \nabla \varphi (x)^T \nabla x = 0 \\
    \Rightarrow &\nabla \varphi (x)^T \nabla x = - \nabla^2 \varphi (x) x \\
    \Rightarrow &\nabla \varphi (x)^T I_d = - \nabla^2 \varphi (x) x \\
    \Rightarrow &\nabla \varphi (x) = - \nabla^2 \varphi (x) x \qquad \square
\end{align*}

\textcolor{blue}{\underline{Claim: $\langle x, \nabla^2 \varphi (x) x \rangle = 1$ } }
\bigskip

From the previous part, we have:

\begin{align*}
    \nabla \varphi (x) = - \nabla^2 \varphi (x) x
\end{align*}

Multiply both sides by $x^T$, we have:

\begin{align*}
    x^T \nabla \varphi (x) = - x^T \nabla^2 \varphi (x) x
\end{align*}

Which is equivalent to the following by using $\langle \nabla \varphi (x), x \rangle = - 1$:

\begin{align*}
    \langle x, \nabla^2 \varphi (x) x \rangle = - \langle x, \nabla \varphi (x) \rangle = (-1) \times (-1) = 1 \qquad \square
\end{align*}





\end{document}